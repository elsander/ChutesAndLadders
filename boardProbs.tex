\documentclass{article}

\begin{document}

\section{Square probabilities}
Chutes and Ladders runs as a Markov chain, meaning that the probability of landing on
one square depends only on where you were at the previous step. I am
using the board configuration in basicboard.png.

I want the probabilities of landing on a square to add to 100 at each
step, so I need to choose what counts as "landing" on a square. Here I
make the arbitrary choice that the square you hit before going on a
chute or ladder is the square you "landed" on. Then if you want to
calculate the probabilities such that landing is where you end up at
the end of your turn, you can add the probability of the
``source'' square to the probability of the ``sink''
square to get that formulation of the problem.

Then:

$P(1) = P(2) = P(3) = P(4) = P(5) = P(6) = 1/6$ on turn 1, 0 otw

In general:

\begin{equation}
P(n_t) = \frac{1}{6}\sum_{i=1}^6{p((n-i)_{t-1})}
\end{equation}

\noindent when there are no chutes or ladders. When there are chutes or ladders,
the sources and sinks of the chutes/ladders must be included when
calculating the conditional probabilities. For example, the
probability of landing on square 18 becomes:

\begin{equation}
P(18_t) = \frac{1}{6}P(17_{t-1}) + \frac{1}{6}P(16_{t-1}) +
\frac{1}{6}P(6_{t-1}) + \frac{1}{6}P(15_{t-1}) +
\frac{1}{6}P(14_{t-1}) + \frac{1}{6}P(4_{t-1}) +
\frac{1}{6}P(13_{t-1}) + \frac{1}{6}P(12_{t-1})
\end{equation}

\noindent Thus the probability of landing on squares just past sinks
increases, and the probability of landing on squares just past sources
decreases. A plot of the probability of landing on each square over
120 steps (each line is a square) is in
ProbabilitiesBySquare.pdf. There is also a version without the
accumulated probability in square 100 to make it more readable. The
same data for 100 time steps can be found in
ProbabilitiesBySquareByStep.txt.

\section{Game Length}

The probability of being at each square at a given time is a
probability distribution conditional over the time $t$. Then the
expected game length is an expectation over these conditional
probability distributions:

\begin{equation}
E(t) = \sum_{t=1}^{\infty}{tP(100|t)} = 1P(100|t=1) + 2P(100|t=2) + \dots
\end{equation}

\noindent Each conditional probability is a sum over all combinations
of squares that could have gotten the player to the finish in
precisely the given number of steps. Since it is a Markov chain,
though, it is possible to simply keep the probabilities of all squares
from the previous time step. Then, in principle, it is possible to
calculate the probability of winning at every time step; in practice,
we can approximate the expectation well by calculating it for a very
large number of time steps. Games are generally not very long, so this
will not be a significant underestimate of the expectation. Based on
my code over 100,000 time steps (at which point the probabilities
are smaller than precision error), the expected game length is
20.5712. I believe this is incorrect, as simulations show the length
to be closer to 37.23 turns.

\section{Effect of multiple players}

Players do not interact in chutes and ladders. This means that the
length of a multi-player game can be thought of as multiple trials of
a single-player game. Then the length of a 2-player game is equal to
the minimum of two trials of a single-player game, and the problem is
to find the expectation of the minimum of two iid random variables.

Considering the probability mass function of this case with two
players X and Y, we have:

\begin{equation}
P(X \wedge Y win at t) = P(X lose|t)P(Y win|t) + P(X win|t)P(Y lose|t) + P(X win|t)P(Y win|t)
\end{equation}

\begin{equation}
= 2P(100|t)(1-P(100|t))+P(100|t)^2
\end{equation}

\noindent where $P(100|t)$ is the probability of landing on square 100
on turn $t$. Then the expectation is:

\begin{equation}
E(t) = \sum_{t=1}^{\infty}{t(2P(100|t)(1-P(100|t))+P(100|t)^2)}
\end{equation}

\noindent Note that the pmf is related to the binomial
distribution. This fact can be used to extend this problem to $m$
players:

\begin{equation}
P(at least 1 win at t) = \sum_{i=1}^{m}{(m choose i)P(100|t)^i(1-P(100|t))^{m-i}}
\end{equation}

\begin{equation}
= 1 - (1-P(100|t))^m
\end{equation}

\noindent which can be used to calculate the expectation as
usual. Using these equations, I calculate the expected game length to
be 40.996 for 2 players and 61.277 for 3, but the lengths are longer,
not shorter, so there is something wrong with my code, and I was not
able to find it in time. I believe the problem is with the code and
not the math behind it.

I don't have time to fix the code, but I ran simulations in the
background to compare against my analytical results. For these data, I
have the expected game length to be approximately 37.23 turns for one
player (based on N=1,000,000), 25.14 for two players (based on
N=100,000), and 20.91 for three (based on N=100,000).

\begin{equation}
\end{equation}

\end{document}